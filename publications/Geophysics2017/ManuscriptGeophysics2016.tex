\documentclass[manuscript]{geophysics}
% \documentclass[paper,twocolumn,twoside]{geophysics}
% \documentclass[paper]{geophysics}
% \documentclass[manuscript,revised]{geophysics}
%%fakesection ===    PACKAGES & DEFINITIONS    ===

% \usepackage{draftwatermark}
% \SetWatermarkLightness{.95}
% \SetWatermarkFontSize{2cm}
% % \SetWatermarkText{\shortstack[c]{Submitted to \texttt{Geophysics}\\23 Oct 2012}}
% \SetWatermarkText{\shortstack[c]{Re-submitted to \texttt{Geophysics}\\05 Feb 2013}}

%% ~ ADDITIONAL PACKAGES TO GEOPHYSICS.CLS
\DeclareGraphicsExtensions{.pdf,.png,.jpg}
\usepackage{color}
\usepackage{tabularx}
\usepackage{colortbl}
\usepackage{booktabs}
\usepackage[USenglish]{babel}
\usepackage[utf8]{inputenc}
\usepackage{lmodern}
\usepackage[T1]{fontenc}
\usepackage{amssymb, amsmath, amsfonts}
% \usepackage[pdftex, draft]{hyperref}
\usepackage{xspace}                 % For spaces after \newcommand-strings
\usepackage{upquote}

% vvv STUFF NOT FOR SUBMISSION
\usepackage[pdftex, final]{hyperref}
\definecolor{myblue}{rgb}{0,0,.5}
\hypersetup{allcolors=myblue, allbordercolors={0 0 .5},
            pdfborderstyle={/S/U/W .5}, colorlinks=true,}
% ^^^ STUFF NOT FOR SUBMISSION

% Figure directory
\renewcommand{\figdir}{figures}
\usepackage[strings]{underscore}

% \newcommand{\tnt}[1]{\texttt{#1}}
\newcommand{\tnt}[1]{#1}
\newcommand{\fortran}{\tnt{Fortran}\xspace}
\newcommand{\python}{\tnt{Python}\xspace}
\newcommand{\matlab}{\tnt{Matlab}\xspace}
\newcommand{\cc}{\tnt{C}\xspace}
\newcommand{\emmod}{\tnt{EMmod}\xspace}
\newcommand{\iemmod}{\tnt{iEMmod}\xspace}
\newcommand{\empymod}{\tnt{empymod}\xspace}
\newcommand{\fftlog}{\tnt{FFTLog}\xspace}
\newcommand{\dipole}{\tnt{Dipole1D}\xspace}
\newcommand{\green}{\tnt{Green3D}\xspace}
%
% Figure and table widths
%   SEGTex               column width = 250 pt
%   Geophysics           column width = 240 pt (3.33 in)
%   Geophysics       1.3 column width = 312 pt ( 4.33 in)
% \plot[btp]{figure1}{width=240pt}{Diff porosities higher than roughly 40~\%.}
% \plot*[btp]{figure11}{width=\textwidth}{(a) i horizontal resistivities.}
%
\definecolor{MyGray}{gray}{0.85}
\newcolumntype{Y}{>{\columncolor{MyGray}\raggedleft\arraybackslash}r}
\newcolumntype{W}{>{\raggedleft\arraybackslash}r}
\newcolumntype{A}{>{\columncolor{MyGray}\raggedleft\arraybackslash}X}
\newcolumntype{B}{>{\raggedleft\arraybackslash}X}

\hyphenation{iso-tro-pic pe-ne-tra-ting}

\begin{document}

\title{An open-source electromagnetic modeler in Python: empymod}

\renewcommand{\thefootnote}{1}% \fnsymbol{footnote}}

\ms{GEO-2016-????}

\address{Instituto Mexicano del Petróleo,
         Eje Central Lázaro Cárdenas Norte 152,
         Col. San Bartolo Atepehuacan C.P. 07730,
         Ciudad de México, México.
         E-mail: \href{mailto:dieter@werthmuller.org}{Dieter@Werthmuller.org}.}

\author{Dieter Werthmüller\footnotemark[1]}

\footer{}
\lefthead{Werthmüller}
\righthead{Open-source EM modeler in Python}

\maketitle

%%fakesection ===    ABSTRACT    ===
\begin{abstract} % 1-2 sentence(s) each
% 1. Principal objectives and scope of the work
  The free software \empymod combines two earlier presented algorithms in this
  journal, creating a new code written in \python that is faster and leaner.
% 2. Methodology
  The main objective is to present a three-dimensional layered-earth
  electromagnetic modeler with vertical transverse isotropy in an easy
  accessible programming language, both in terms of costs and learning curve,
  and in a collaborative way. The code is hosted in a web-based Git repository
  under a lax permissive license, allowing anyone to use it, even for
  commercial purposes, as well as to contribute to its development.
% 3. Results
  Comparisons show that this code is as precise and faster than the codes it is
  based upon, thanks to the use of different Hankel transforms and the
  throughout vectorization of the calculation.
% 4. Conclusions
  This code might certainly be useful for professionals in the electromagnetic
  area, but I specifically hope this code to be useful for educational
  purposes.
\end{abstract}

\section{Introduction}

% 1.1 CSEM in hydrocarbon exploration
The potential of electromagnetic methods for the detection of hydrocarbon
reservoirs is known for some decades, see for instance \cite{PIEEE.89.Nekut} or
\cite{B.SEG.91.Chave}. More recent, good overviews of the methodology and its
applications are give by \cite{SG.05.Edwards} and \cite{IEEE.12.Ziolkowksi}.
Whereas in the early days everything happened in idealized, isotropic one
dimensional (1D) earth models, it is generally agreed that the often complex
geology of hydrocarbon reservoirs requires two dimensional (2D) and three
dimensional (3D), anisotropic forward modelers. However, 1D models are still
very important, not last because they are very fast. Many problems can be
simplified to and henceforth solved with 1D models. More importantly, 1D allows
to study single, isolated effects to the electromagnetic field, which is a
crucial foundation in understanding the electromagnetic field behaviour and a
necessity for understanding the phenomena at higher dimensions. 1D models are
furthermore often used in inversion routines of higher dimensions, for instance
to generate a starting model or embed a 3D body in a 1D background.

% 1.2 Existing 1D solutions
Solutions for the electromagnetic fields in a layered-earth model have been
solved and published extensively using different approaches. The importance and
wide\-spread use of 1D models is shown in the continuous stream of publications
in this area, even in recent years. \cite{GJI.07.Loseth} solved the problem
with the scattering matrix formulation for a 1D earth with general anisotropy,
spanning the frequency range from controlled-source electromagnetics (CSEM) to
ground-pe\-ne\-tra\-ting radar (GPR).  \cite{GJI.09.Chave} presented a solution
in terms of independent and unique transverse electric (TE) and tangential
magnetic (TM) modes for the electrical isotropic case using the diffusive
approximation (without displacement currents, valid for low frequencies such as
in CSEM). \cite{GEO.09.Key} demonstrated why 1D models still matter by testing,
for instance, the benefit of additional frequencies in an inversion routine.
Key follows and extends the magnetic vector potential approach by
\cite{B.AP.82.Wait} for forward modeling, using the isotropic, diffusive
low-frequency approach.  \cite{GEO.15.Hunziker} obtained the electromagnetic
field in a layered earth with vertical transverse iso\-tro\-py (VTI) by solving
two equivalent scalar equations with a scalar global reflection coefficient.
All of the four citations regarding 1D solutions have quite extensive reference
lists about the history of 1D forward modeling, the last one featuring an
interesting review of the history of 1D electromagnetic derivations spanning
almost 200 years.

% 1.3 Hankel transform
A crucial as well as very interesting part of electromagnetic modeling is,
once one moves from the theoretical derivation to the numerical implementation,
the Hankel transform involved in the transformation from the
wave\-num\-ber-fre\-que\-ncy domain to the space-frequency domain. The use of
digital filters is quite common in geophysics, known as the fast Hankel
transform method (FHT), as introduced by \cite{GP.71.Gosh}, and popularized by
\cite{TRP.75.Anderson, GEO.79.Anderson, TMS.82.Anderson} thanks to his freely
available \fortran routines. Naturally, standard quadrature can be used as well
for the Hankel transform, see for instance \cite{GEO.83.Chave}, and hybrid
routines using both methods were published by \cite{GEO.84.Anderson,
GEO.89.Anderson}. The topic got picked up again recently with new filters being
published by \cite{GP.07.Kong} and \cite{GEO.12.Key}.

% 1.4 1D codes and licenses
In terms of freely available and open-source code there are a few examples.
\cite{GEO.09.Key} published his forward modeling and inversion code \dipole,
written in \fortran. The dipole can be placed anywhere in the stack of layers
and can have arbitrary orientation and dip. \dipole computes the electric and
magnetic fields to an electric source, using the FHT method. \cite{GEO.12.Key}
published additional code with his introduction of the
qua\-dra\-ture-with-ex\-tra\-po\-la\-tion method (QWE) and its comparison to
the FHT method, for which he translated some of the \dipole-\fortran code to
\matlab. \cite{GEO.15.Hunziker} published their code \emmod (\fortran and \cc),
in which the source and receiver can also be placed anywhere in the stack of
layers. They use a 61\,pt Gauss-Kronrod quadrature for the Hankel transform.

% 1.5 Claim
With \empymod I present a 1D forward modeling code that is based on
\cite{GEO.15.Hunziker} for the wavenumber-frequency domain calculation, and on
\cite{GEO.12.Key} for the Hankel and Fourier transforms. As I will therefore
refer to these publications quite a lot, I use the name Hun15 to denote
\cite{GEO.15.Hunziker}, and the names Key09 and Key12 to denote
\cite{GEO.09.Key, GEO.12.Key}. To denote the FHT filters as published by Key09
and Key12 I will add the corresponding filter-size, e.g. Key09-401.
In addition, \empymod includes the logarithmic fast Fourier transform \fftlog
of \cite{RAS.00.Hamilton} as a third Fourier transform possibility to FHT and
QWE for the frequency-to-time transformation.

To my knowledge, \empymod is the first code that is freely available which
calculates the full wavefield for a layered-earth model with vertical isotropy
and has both quadrature and filters built in. This makes \empymod ideal not
just for comparison studies of the two methods, but also to build hybrid
inversion schemes.  Further advantages of \empymod are:

\textbf{(1) \python:} \python is a modern, cross-platform, free and open-source
programming language. With its scientific libraries, mainly the numeric and
scientific modules \texttt{NumPy} and \texttt{SciPy}, it creates an extremely
powerful numerical calculation stack.

\textbf{(2) Libre (free and open-source):} The code is published under the lax
permissive \emph{Apache Version 2.0 license}, which makes it available to
everyone, even for commercial purposes. It therefore might prove to be valuable
for students and professionals alike. Not only the code is free, but also the
underlying platform (\python), contrary to, for instance, \matlab. As such it
is ideal for reproducible research.

\textbf{(3) Lean:} The codebase is very lean. It is built from scratch, and it
therefore does not suffer the problems that sometimes come with the organic
growth of a codebase. The code follows as much as possible the \emph{DRY}
coding paradigm, \emph{Don't Repeat Yourself}.

\textbf{(4) Fast:} The code is vectorized as much as possible, which improves
computation. The most time-consuming calculations have furthermore a flag to
run parallelized. However, on a larger scale such as an inversion routine one
would probably want to parallelize the kernel calls instead of the kernel
itself. Even though \python is an interpreted language it can be very fast, as
the underlying routines are in \fortran or in \cc (using for instance the
BLAS/Lapack libraries), and \python is merely the glue. The comparisons show
that \empymod is as fast as the existing codes.

\textbf{(5) Community:} The code is hosted on GitHub, which makes it easy for
anyone to improve and contribute to the code, and will allow \empymod to
grow:\\
\url{https://github.com/prisae/empymod}.

\textbf{(6) Well documented:} The code is extensively documented, and large
parts of it are extracted using automated tools (Sphinx) to create a manual:\\
\url{https://empymod.readthedocs.io}.

These points make this a worthwhile extension to the existing codes from Hun15
and Key12. The existing codes are written in \fortran, which requires much more
time than \python to get started for students or to develop for anyone, or
\matlab, which is a proprietary language. And the existing codes are in static
repositories where one can only download the code, which makes interaction,
contribution, or bug filing more difficult.

After introducing the code in the first part I will present some comparisons by
reproducing results from the publications this code is based upon: First a
comparison to the analytical half-space solution, followed by a comparison to
\emmod by Hun15, and finally a comparison to some results presented by Key12.

% Add a note about the outcome? Speed/Precision?

\section{About the code}

The code consists of 5 files; these are 3 core modules plus \emph{utils}, which
contains input checks and other utilities, and \emph{filters}, containing the
FHT filter coefficients. The three core routines are: (1) \emph{kernel}, where
the wavenumber-domain calculation is carried out; (2) \emph{transform}, where
the Hankel and Fourier transforms are computed; and (3) \emph{model}, which
contains the actual modeling routines for end users. As of now there are two
main modeling routines implemented, \emph{frequency} and \emph{time}, with
which one can calculate the frequency- and time-domain responses for electric
or magnetic point sources and receivers, directed along the three principal
axis $x, y,$ and $z$. More modeling routines can easily be added to
\emph{model}, such as finite bipoles or arbitrary source and receiver
directions, as they do not affect the core of the calculation, hence not
\emph{kernel} nor \emph{transform}.

The wavenumber-domain calculation in \emph{kernel} follows Hun15, and
calculates as such the complete wavefield for a layered VTI model, where the
code makes no assumptions about the model. Source and receiver can be placed
anywhere in the model. Depths, frequencies, and source-receiver configuration
have to be defined, and each layer is characterized with its horizontal
resistivity $\rho_h$, its electrical anisotropy $\lambda$, where $\lambda =
\sqrt{\rho_v/\rho_h}$, its horizontal and vertical magnetic permeabilities
$\mu_h$ and $\mu_v$, and the horizontal and vertical electric permittivities
$\epsilon_h$ and $\epsilon_v$. The main differences between \emmod and \empymod
are the programming language and the Hankel transform, and a few additional
fundamental differences: \emmod uses 2nd order Bessel functions of the first
kind $J_2$. Published FHT filters generally provide coefficients for 0th and
1st order Bessel functions $J_0, J_1$ only. Therefore, the recurrence relation
%
\begin{equation}
  J_2(kr) = \frac{2}{kr}J_1(kr) - J_0(kr)
  \label{eq:j2}
\end{equation}
%
is used, where $k$ and $r$ are the wavenumber and space-domain parameters,
respectively. Another difference is vectorization: \emmod carries out the
calculation in the wavenumber-domain by looping over frequencies, offsets, and
wavenumbers. This is carried out in a single calculation without looping in
\empymod. Another difference is that \empymod is much leaner than \emmod.
\emmod grew organically, as confirmed by the author, adding more and more
features, where \empymod was designed with all 36 source-receiver
configurations from the beginning. As an example, the full-space solution in
\empymod is one function (107 lines of code) for all source-receiver
configurations, whereas in \emmod it is split into no less than 16 functions
(364 lines of code). This should help to maintain the code easier, and it
should also be easier for new contributors to read into the code.

The Hankel and Fourier transforms in \emph{transform} follow Key12, with a few
changes. The most important ones regarding speed is vectorization and a splined
version for the filter method, in addition to the traditional lagged version:
The lagged version samples the wavenumber from the minimum to the maximum
required wavenumber given the required offsets and the chosen filter base with
the spacing as defined by the filter. It subsequently carries out the Hankel
transform, and interpolates for the required offsets in the space domain. The
splined version, on the other hand, uses a user-specified number of values per
decade from the minimum to the maximum wavenumber. It then interpolates for the
required wavenumber values, and does the Hankel transform afterwards. The
lagged version is \emph{very} fast. However, with the splined version a
speed-up can be achieved in comparison with the original FHT at higher
precision if compared with the lagged version. All filters published by Key09
and Key12 are included in \empymod, which includes Key's own filters as well as
the filters by \cite{TMS.82.Anderson} and by \cite{GP.07.Kong}.

The most import part of Key12 is the introduction of a new quadrature algorithm
to geophysics, named qua\-dra\-ture-with-ex\-tra\-po\-la\-tion (QWE). QWE is a
fast quadrature method using the Shanks transformation \citep{JMP.55.Shanks}
computed with Wynn's epsilon algorithm \citep{MC.56.Wynn}. The advantage of
quadrature over filters is the ability to estimate the error. QWE continues
until the absolute error, estimated by the difference of subsequent iterations
$n$, satisfies the inequality
%
\begin{equation}
  |S^*_n-S^*_{n-1}| \le \varepsilon_r|S^*_n| + \varepsilon_a\ ,
  \label{eq:err}
\end{equation}
%
where $S^*$ is the extrapolated result, and $\varepsilon_r, \varepsilon_a$ are
the relative and absolute tolerance, respectively.

Whereas for the Hankel transform the two methods QWE and FHT are implemented,
for the Fourier transform there are three methods implemented: QWE, FHT with
the sine and cosine filters, and \fftlog \citep{RAS.00.Hamilton}. The
logarithmic fast Fourier transform \fftlog is ideal for this operation, as
generally a wide range of frequencies is required to go from frequency to time
domain. The \fftlog can yield faster results than the QWE method and more
precise than the FHT results, specifically for land impulse responses at early
times.

In addition to the splined version of the QWE and the splined and lagged
versions of the FHT, all time-consuming calculations are set up to be able to
run in parallel, with the help of the \python-module \tnt{numexpr}.  This can
significantly speed up calculations if you run big models with many layers and
for many offsets and frequencies. However, if you include \empymod in an
inversion scheme then it might be better to parallelize the calls to \empymod,
instead of \empymod itself.

It is important to note that calculations in the wavenumber domain depend
\emph{only} on offset $r = \sqrt{x^2+y^2}$; the scaling factor, which depends
on the angle $\varphi = \rm{atan2}(y, x)$, is multiplied afterwards. In order
to calculate for instance a circle around the source, the kernel has to be
called only once, and subsequently scaled by the angle-dependent factor for
each source-receiver pair. This makes the splined version very powerful, as it
can be used for irregularly distributed data.

The code distributed on GitHub contains \tnt{Jupyter Notebooks} to reproduce
the results in this article, as well as some basic tests and benchmarks, and
the \LaTeX-source of the article itself.

The run-time comparisons tests were run on a Lenovo ThinkCentre running Ubuntu
16.04 64-bit, with 8\,GB of memory and an Intel Core i7-4770 CPU @ 3.40GHz x 8
(4 cores with hyper-threading). Comparing run times is always a difficult task,
and between different languages even more, here between \python, \fortran, and
\matlab. However, they do serve as comparison.  I used the
\matlab-\texttt{timing} and \python-\texttt{timeit} functions, which behave
very similar. It is probably expected today, but still worth mentioning, that
the calculation is carried out in double precision.

\section{Comparison to analytical half-space solution}

\cite{PIER.10.Slob} published analytical frequency- and time-domain solutions
for the diffusive electric field in a VTI half-space. As an example, I use the
same model as Hun15 in their Figures~1 and~2: A half-space with horizontal
resistivity $\rho_h = 1/3\,\Omega$\,m, anisotropy $\lambda = \sqrt{10}$, and
for frequency $f = 0.5\,$Hz. The source is located horizontally at the origin
at a depth of 150\,m, and the depth of the receivers is 200\,m. The field is
calculated on a regular grid with a spacing of 10\,m.
Figure~\ref{fig:analytical} shows the analytical amplitude and phase results
for the first quadrant (the other quadrants are simply symmetric copies of it).
%
\plot*[btp]{analytical}{width=\textwidth}{Analytical solution for a
  half-space with $\rho_h = 1/3\,\Omega$\,m, $\lambda = \sqrt{10}$, $f =
  0.5\,$Hz, $z_s = 150\,$m, and $z_r = 200\,$m, calculated on a regular grid
  with a spacing of 10 meters; (a) amplitude and (b) phase.}
%
The figure shows the result for x-directed electric source and receiver
dipoles ($G^\mathrm{ee}_{xx}$), other configurations yield similar results (the
choice is based on the insight that this is the most interesting of all
electric source to electric receiver fields, as can be seen in Hun15 Figures~1
and~2).

The error of the amplitude is shown in Figure~\ref{fig:onederror-amplitude} for
different settings regarding the Hankel transform: (a) for a 51\,pt QWE with
relative tolerance $\varepsilon_r$ and absolute tolerance $\varepsilon_a$ of
$10^{-12}$ and $10^{-30}$, respectively; (b) for a 21\,pt QWE with
$\varepsilon_r = 10^{-8}$ and $\varepsilon_a = 10^{-30}$; (c) for a 15\,pt QWE
with $\varepsilon_r = 10^{-8}$ and $\varepsilon_a = 10^{-18}$; (d) using the
standard FHT method with the filter Key09-401; (e) using the splined FHT of the
same filter with 40 points per decade; and (f) using the lagged FHT of the same
filter.
%
\plot*[btp]{onederror-amplitude}{width=.92\textwidth}{Error levels for different
  Hankel transform settings, compared with the analytical solution in Figure~1.
  The high precision QWE (a) and the standard FHT (d) have very low error
  levels, generally in the order of $10^{-6}$\,\% or less. The lagged FHT has
  much higher error levels, and the effects of interpolation can be seen
  clearly in the ring-like structure. However, almost the entire error is below
  0.1\,\% ($10^{-1}$\,\%), only the dark black parts have an error of 1\,\% or
  slightly more.}
%
The results from the high precision QWE (a) and the standard FHT (d) are almost
identical, even though they use completely different Hankel transforms. This
can be seen better in Figure~\ref{fig:onederror-amp1d}, which shows a
cross-section through subplots (a), (c), (d), and (f) of
Figure~\ref{fig:onederror-amplitude} at a crossline offset of 4\,km.
%
\plot[btp]{onederror-amp1d}{width=.5\textwidth}{Relative percentage error for
  crossline offset of 4\,km as shown in Figure~\ref{fig:onederror-amplitude}
  (a), (c), (d), and (f).}
%
The error is, in this case, not due to the method used for the Hankel transform,
but rather to the limitations in computation: either because we reach the
numerical noise level, or because of distinct differences between the
analytical solution, which uses the diffusive approximation, and the numerical
code, which calculates the complete field. If the QWE is calculated for lower
tolerance levels, as shown in (b) and (c), or the FHT is used with
interpolation as in (e) and (f), artefacts seem to arise from the Hankel
transform and the interpolation.

The error of the phase is shown in Figure~\ref{fig:onederror-phase}, with the
same conclusion.
%
\plot*[btp]{onederror-phase}{width=.92\textwidth}{Same as
  Figure~\ref{fig:onederror-amplitude}, but for the phase. Interesting is to
  see that interpolation in the wavenumber domain (e) yields slightly different
  patterns than interpolation in space domain (f).}
%

\section{Comparison to Hunziker et al. (2015)}

Hun15 derive the electromagnetic fields in astoundingly simple equations for
all 36 possible source-receiver combinations (electric and magnetic sources and
receivers in three directions $x,y,z$) by finding the solution for the
vertical electric field and then applying the duality principle and reciprocity
to derive all components. The corresponding code \emmod is published on the SEG
website, and in \cite{GEO.16.Hunziker} they published with \iemmod an inversion
routine for it.

\emmod is written in \fortran and \cc. On execution, all parameters are
provided in an input file, and the resulting responses are written to an output
file.  The calculation is carried out for a regular grid in the space domain,
with at least two points in each direction. The code calculates in loops the
solution for all wavenumbers for the first quadrant, carries out the
wavenumber-to-space transformation, and then copies the result for the other
four quadrants, writing everything to the output file. This approach works very
well and even for millions of cells. However, the usage is probably more
academic. When it comes to actual measurements one deals with dozens to at most
hundreds of offsets, and usually on an irregular grid. \dipole and \empymod
work more along the practical approach.

Figure~\ref{fig:emmod-HS} shows the error of amplitude and phase for \emmod.
On the left side with a colorscale from $10^0$ to $10^2$\,\% as used in Hun15,
on the right side with a colorscale from $10^{-8}$ to $10^0$\,\% as used in
Figures~\ref{fig:onederror-amplitude} and~\ref{fig:onederror-phase}. Note that
the error calculation in Hun15 was done slightly different. Here I show the
relative error between the analytical result and the result from \emmod, where
Hun15 shows the relative error between $\log_{10}$(analytical result) and
$\log_{10}$(result from \emmod).
%
\plot*[btp]{emmod-HS}{width=.7\textwidth}{Error of \emmod for the half-space
  model in Figure~1. On the left side with the same colorscale as in Hun15, on
  the right side with the colorscale as in Figures~2 and~3.}
%
Figure~\ref{fig:emmod-HS} shows a few interesting points. The responses from
\emmod in this example have significantly less precision than the results from
\empymod. This does not mean in any way that \emmod is less precise, as \emmod
can be adjusted to yield much preciser results. However, this would
significantly increase the run time, so it has to be kept in mind for the run
time comparison. The important point is that \emmod does \emph{always} do an
interpolation in the space-frequency domain, by default, a linear
interpolation.  No interpolation is used in \empymod, unless one specifies it
to speed up the calculations (splined QWE or splined and lagged FHT options),
which will therefore yield preciser results. This can be seen very nicely in
the results.  Figures~\ref{fig:emmod-HS} (b) and (d) show white circles where
the result is very precise. These are the offsets close to those where the
fields were actually calculated. The black bands in-between are the areas where
the result was interpolated. It also shows that the spacing between calculated
offsets increases with increasing offsets. The error patterns from \empymod in
Figures~\ref{fig:onederror-amplitude} and \ref{fig:onederror-phase} show
generally a different pattern, displaying where the code hits the numerical
accuracy or differences between the analytical, diffusive solution and the
numerical, full wavefield solution. It is only for the splined and lagged
versions that one sees the same, circular pattern appearing, which are due to
interpolation.

Table~\ref{tbl:analytical} shows run times for these models. A few notes worth
mentioning: \empymod carries out a lot of input checks, as it is quite
forgiving in what you input. \emmod, on the other hand, reads the input file
and writes the result to an output file, and copies the first quadrant to the
other three. Both have therefore some overhead, and the comparison is not
strictly 1:1. For the comparison the same model was used as shown above, on a
regular grid with a spacing of 100\,m.  On the test-machine the standard,
vectorized, QWE and FHT \empymod would run into memory issues for denser
spacing, hence arrays of more than some 10'000s of offsets.  The splined and
lagged versions of \emmod could handle it, however.
%
\tabl[btp]{analytical}{Run times for the half-space model shown in Figure~1, on
  a regular grid with spacing of 100\,m, for \emmod and different Hankel
  transform settings of \empymod (times in milliseconds).}{
  \centering
  \begin{tabularx}{240pt}{rAAABBB}
  \toprule%
  EMmod & \multicolumn{3}{c}{QWE} & \multicolumn{3}{c}{FHT} \\
        & 1 & 2 & 3               & 1 & 2 & 3 \\
  \midrule%
   5030 & 7920 & 2380 & 1180 & 3030 & 1640 & 6 \\
  \bottomrule%
  \end{tabularx}
}
%
QWE becomes faster for decreasing precision, not surprisingly, and the splined
and lagged version of FHT are faster than the standard version. The lagged FHT
is the fastest by quite a margin, and about 1/3 of that time is from the input
checks, so it takes only about 4\,ms to calculate the 11'025 offsets.
What is interesting here is that the lagged FHT is still more precise than
\emmod with the given settings, which means a speed-up of a factor 1000 for the
same result.

Many 1D CSEM codes use the diffusive approximation that is valid for low
frequencies. The appealing part of the derivation of Hun15 is that it models
the complete wavefield, hence it is valid for high frequencies and therefore
wave-phenomena. As an extreme case, Hun15 show a 1D example for
ground-penetrating radar with a center frequency of 250\,MHz. To do so, 2048
frequencies in the range of 1\,MHz to 2048\,MHz are calculated for the Fast
Fourier transform from frequency to time domain.  Figure~\ref{fig:gpr} shows
the three results for \emmod, \empymod with a 21\,pt QWE and relative and
absolute tolerance of $10^{-10}$ and $10^{-18}$, respectively, and \empymod
with FHT with the filter Key09-401, together with the analytical solution for
the arrival of the direct wave (red), the wave refracted at the surface (cyan),
and the wave reflected at the subsurface interface (magenta). For the results
with \empymod I used the regular FFT as Hun15, not \fftlog.

\emmod does clearly do the best job. QWE has troubles at very short offsets (<
0.1\,m), and a \emph{noisy triangle} expanding from the origin. This triangle
could be narrowed by increasing the precision of the QWE, at the cost of
computation time. However, I was not able to completely get rid of it with QWE.
FHT shows the first arrivals well, however, afterwards the result becomes
extremely noisy.
%
\plot*[btp]{gpr}{width=1\textwidth}{GPR example for (a) \emmod, (b) \empymod
  with 21\,pt QWE for relative and absolute tolerance of $10^{-10}$ and
  $10^{-18}$, respectively, and (c) \empymod with FHT (Key09-401); run times
  are given in the subplot titles.}
%
Interesting are the calculation times for these three results. \emmod took more
than 32 hours to calculate, \empymod with QWE about 34 minutes, and \empymod
with FHT a bit over 4 minutes. Surprising is how good the FHT result is, given
that the filter was designed for CSEM data, hence frequencies 6 to 9 orders of
magnitudes smaller. One could implement the Gauss-Kronrod quadrature into
\empymod, and see how this would compare to \emmod for GPR data in terms of
speed and precision. But, in any case, \emmod/\empymod are not optimized for
nor intended to model GPR data, and calculations in time-domain will be much
faster and more precise. It is nevertheless an interesting proof of concept and
shows that \emmod/\empymod model the entire EM field.

\section{Comparison to Key (2009, 2012)}

Key12 not only introduces QWE to geophysics, but also compares QWE to FHT for
various filters, some of which were specifically created for the comparison. In
order to calculate the models he translates parts of \dipole (Key09) from
\fortran to \matlab. The models are therefore isotropic and use the diffusive
approximation. He summarizes succinct and clear the FHT method and outlines the
QWE method, and made the codes available from the SEG website.

The inclusion of the filter method FHT and the quadrature method QWE in
\empymod follows Key12, with one important exception: vectorization. The
\matlab code of Key12 loops over frequencies, offsets, and wavenumbers; the
code was developed to compare the different hankel transforms, not with speed
in mind.  In \empymod both methods are vectorized, hence various frequencies
and offsets can be calculated for all required wavenumbers in one calculation,
which changes the conclusion drawn in Key12 comparing the speed of QWE and FHT.
The vectorized approach is much faster but is limited by the memory of the
computer. If the model becomes too big, which depends on numbers of layers,
frequencies, offsets, and wavenumbers, the calculation has to be carried out by
looping over frequencies or offsets or both; the code never loops over
wavenumbers.

In Key12, the standard QWE is always faster than the standard FHT. In the
vectorized version of \empymod, the standard QWE \emph{can} be faster than the
standard FHT if the model is big and many offsets are required, but often it is
slower. Furthermore, the lagged FHT is generally much faster than the QWE
method, splined or not.

Table~\ref{tbl:runtimes} lists the run times for exactly the same models as
Key12 compared in his Table~1. In this comparison, a 9\,pt QWE is compared to a
201\,pt (Key12-201) and a 801\,pt filter (Key12-801), where the absolute and
relative tolerance are set so that the QWE achieves a similar error-level as
the filters ($\varepsilon_r = 10^{-6}$ for the standard QWE and $10^{-2}$ for
the splined version, $\varepsilon_a = 10^{-24}$ in both cases). In order to
make a fair comparison I re-run the script from Key12, and the run times on the
test-machine are roughly twice as fast as in the original paper. Next to it are
the results from \empymod. It can be seen from the results that \empymod is
significantly faster.%
%
\tabl[btp]{runtimes}%
  {Run times comparing Key12 (Key) and \empymod (Wer) for the same cases as in
  Key12, Table~1 (in milliseconds).}{
  \centering
\begin{tabularx}{1.05\textwidth}{ccYYWWYYWWYYWW}
  \toprule
  %
  & & \multicolumn{4}{c}{QWE: 9\,pt}&\multicolumn{4}{c}{FHT: 201\,pt filter}&
  \multicolumn{4}{c}{FHT: 801\,pt filter} \\
  %
  \cmidrule(rl){3-6} \cmidrule(rl){7-10} \cmidrule(rl){11-14}
  %
  & &
  \multicolumn{2}{>{\columncolor{MyGray}\arraybackslash}c}{Normal} &
  \multicolumn{2}{c}{Splined} &
  \multicolumn{2}{>{\columncolor{MyGray}\arraybackslash}c}{Normal} &
  \multicolumn{2}{c}{Lagged} &
  \multicolumn{2}{>{\columncolor{MyGray}\arraybackslash}c}{Normal} &
  \multicolumn{2}{c}{Lagged} \\
  %
  Lay& Off& Wer& Key& Wer& Key& Wer&  Key& Wer& Key& Wer&  Key& Wer& Key\\
  \midrule
  %
    5&   1&   7&  16&   3&  27&   1&   19&   1&  19&   2&   75&   2&  75\\
    5&   5&  14&  69&   5&  29&   3&   92&   2&  24&   7&  364&   3&  81\\
    5&  21&  20& 302&   7&  37&   8&  383&   2&  25&  25& 1520&   3&  83\\
    5&  81&  39&1152&  15&  61&  29& 1475&   2&  28&  97& 5813&   3&  85\\
    5& 321& 104&4559&  49& 163& 122& 5817&   2&  40& 428&22966&   3&  98\\
  \midrule
  100&  21& 136& 458&  18&  50&  91&  602&   9&  37& 301& 2377&  19& 125\\
  100&  81& 325&1769&  25&  75& 339& 2300&   9&  40&1211& 9084&  19& 128\\
  100& 321&1041&7084&  59& 175&1380& 9093&   9&  52&5338&36102&  19& 142\\
  %
  \bottomrule
\end{tabularx}}%
%
Important to note is, however, not the absolute speed of \empymod, but the
difference between the various Hankel transform methods. The standard QWE
method, for a similar error level as FHT, is faster than the standard FHT mode
only for many offsets. More importantly, the lagged FHT is generally much
faster than any QWE. QWE is very useful to check the error level of a result.
If many kernel evaluations have to be carried out, and speed is of importance,
the lagged FHT is the preferred method. As Key12 loops over each wavenumber,
the difference between the vectorized and non-vectorized version can be best
seen in the long 801\,pt filter.

These results regarding speed of calculation do not change the fact that the
advantage of the QWE method is to have an estimate of the error. However, the
filters designed for CSEM problems seem to be very good at solving them, as can
be seen in the low error levels in Figures~\ref{fig:onederror-amplitude}
and~\ref{fig:onederror-phase} or in Key12.


Table~\ref{tbl:runtimes2} lists the run times for the same model as before,
but for \empymod and \dipole using the FHT method with the filter Key09-201,
the default filter in \dipole. For the regular versions (-) \empymod and
\dipole perform very similar; \empymod is a little faster for the smaller
tests, \dipole is slightly faster for the bigger tests. It can be seen from the
results that the parallel optimisation (par) in \empymod can result in a
significant speed-up.  If the lagged convolution optimisation is chosen,
\empymod is significantly faster than \dipole. In \empymod, additional offsets
come at basically no cost in the lagged version, whereas \dipole still uses
more time to calculate more offsets. This is most likely due to the writing of
the output file, that becomes bigger with bigger offsets.
%
\tabl[btp]{runtimes2}
  {Run times comparing \dipole (Key09) and \empymod, using the filter Key09-201
  (default in \dipole; times in milliseconds). Optimisation: [-] None,
  [par] parallel, [spl] lagged FHT.}{ \centering
  \begin{tabularx}{.5\textwidth}{ccAAABB}
  \toprule
  %
     &    & \multicolumn{3}{c}{\empymod} & \multicolumn{2}{c}{\dipole}\\
  %
  \cmidrule(rl){3-5} \cmidrule(rl){6-7}
  %
  Lay& Off&  - & par & spl &   - & spl\\
  \midrule
  %
    5&   1&    1&   2& 1&    4&  4\\
    5&   5&    3&   3& 2&    6&  5\\
    5&  21&    8&   5& 2&   12&  8\\
    5&  81&   30&  10& 2&   36& 18\\
    5& 321&  123&  34& 2&  128& 56\\
  \midrule
  100&  21&   90&  53& 9&   90& 14\\
  100&  81&  340& 107& 9&  331& 23\\
  100& 321& 1423& 312& 9& 1303& 61\\
  %
  \bottomrule
\end{tabularx}}%
%
\dipole was compiled using the free and open-source \emph{gfortran} compiler.
Compiling it with the (proprietary) \emph{ifort} compiler might result in
slightly faster run times.

\section{Conclusions}

The presented code \empymod is a fast, lean, open-source electromagnetic
modeler, well documented and hosted in a way that makes collaboration easy.
It can model the full wavefield of a 3D source in a VTI layered-earth.
Additional to the obvious application of modeling and inversion of
electromagnetic data, \empymod can be used to investigate into the differences
between filters and quadrature for full-wavefield electromagnetic calculations
over a wide range of frequencies. There are not many full-wavefield codes
openly available and, as such, \empymod with QWE and FHT is an important
addition to \emmod, which uses a 61\,pt Gauss-Kronrod quadrature.

Outside of the traditional scope I think \empymod can be very educative for
someone who is just starting to get interested in electromagnetic modeling or
even just in Hankel transforms, hence for educational purpose: it is brilliant
for students, specifically as \python is becoming more and more widespread in
academia, and the number of Universities that teach geophysicist \python
(mostly instead of \matlab) is growing annually. \python, like \matlab, has the
advantage of very fast developing times, and considerably lower entry barriers
than \cc or \fortran for beginners.

There are many possibilities to improve \empymod, or to add additional
functionalities, as in any software (this is why it is important to keep
open-source code in version-controlled repositories such as GitHub instead in
static repositories). Possible additions are more modeling routines, such as
arbitrary source and receiver dipole lengths, arbitrary source and receiver
rotations, or variable receiver depths within one calculation. Other features
to include could be further Hankel and Fourier transforms, such as the
Gauss-Kronrod as in \emmod. The benchmarks and tests included in the code could
also be improved and extended. By the time this article is published some of
them might already be implemented by me or by the community.

\section{Acknowledgment}

I would like to thank the \emph{Consejo Nacional de Ciencia y Tecnología},
México (CONACYT) for funding this postdoc, the \emph{Instituto Mexicano del
Petróleo} (IMP) for allowing me to publish the code under a free software
license, and the entire electromagnetic research group of Aleksandr Mousatov at
the IMP for fruitful discussions.  I owe a special thanks to Jürg Hunziker for
answering all my questions regarding his code and publication, and to both Jürg
Hunziker and Kerry Key for feedback regarding this manuscript that greatly
improved the article.

% REFERENCES
\bibliographystyle{seg}
\begin{thebibliography}{}
\itemsep0pt

\bibitem[Anderson, 1975]{TRP.75.Anderson}
Anderson, W.~L.,  1975, Improved digital filters for evaluating {F}ourier and
  {H}ankel transform integrals: Technical report, U.S. Geological Survey.
\newblock
  (\href{https://pubs.er.usgs.gov/publication/70045426}{https://pubs.er.usgs.gov/publication/70045426}).

\bibitem[Anderson, 1979]{GEO.79.Anderson}
--------, 1979, Numerical integration of related {H}ankel transforms of orders
  0 and 1 by adaptive digital filtering: Geophysics, {\bf 44}, 1287--1305.
\newblock (\href{http://dx.doi.org/10.1190/1.1441007}{doi: 10.1190/1.1441007}).

\bibitem[Anderson, 1982]{TMS.82.Anderson}
--------, 1982, Fast {H}ankel transforms using related and lagged convolutions:
  ACM Trans. Math. Softw., {\bf 8}, 344--368.
\newblock (\href{http://doi.acm.org/10.1145/356012.356014}{doi:
  10.1145/356012.356014}).

\bibitem[Anderson, 1984]{GEO.84.Anderson}
--------, 1984, {On: “Numerical integration of related Hankel transforms by
  quadrature and continued fraction expansion” by \cite{GEO.83.Chave}}:
  Geophysics, {\bf 49}, 1811--1812.
\newblock (\href{http://dx.doi.org/10.1190/1.1441595}{doi: 10.1190/1.1441595}).

\bibitem[Anderson, 1989]{GEO.89.Anderson}
--------, 1989, A hybrid fast {H}ankel transform algorithm for electromagnetic
  modeling: Geophysics, {\bf 54}, 263--266.
\newblock (\href{http://dx.doi.org/10.1190/1.1442650}{doi: 10.1190/1.1442650}).

\bibitem[Chave, 1983]{GEO.83.Chave}
Chave, A.~D.,  1983, Numerical integration of related {H}ankel transforms by
  quadrature and continued fraction expansion: Geophysics, {\bf 48},
  1671--1686.
\newblock (\href{http://dx.doi.org/10.1190/1.1441448}{doi: 10.1190/1.1441448}).

\bibitem[Chave, 2009]{GJI.09.Chave}
--------, 2009, On the electromagnetic fields produced by marine frequency
  domain controlled sources: Geophysical Journal International, {\bf 179},
  1429--1457.
\newblock (\href{http://dx.doi.org/10.1111/j.1365-246X.2009.04367.x}{doi:
  10.1111/j.1365-246X.2009.04367.x}).

\bibitem[Chave et~al., 1991]{B.SEG.91.Chave}
Chave, A.~D., S.~C. Constable, and R.~N. Edwards,  1991, Electrical exploration
  methods for the seafloor, {\it in} Electromagnetic Methods In Applied
  Geophysics Vol.\ 2: SEG, Investigations in Geophysics, No.~3, 12,  931--966.
\newblock (\href{http://dx.doi.org/10.1190/1.9781560802686}{doi:
  10.1190/1.9781560802686}).

\bibitem[Edwards, 2005]{SG.05.Edwards}
Edwards, R.~N.,  2005, Marine controlled source electromagnetics: principles,
  methodologies, future commercial applications: Surveys in Geophysics, {\bf
  26}, 675--700.
\newblock (\href{http://dx.doi.org/10.1007/s10712-005-1830-3}{doi:
  10.1007/s10712-005-1830-3}).

\bibitem[Ghosh, 1971]{GP.71.Gosh}
Ghosh, D.~P.,  1971, The application of linear filter theory to the direct
  interpretation of geoelectrical resistivity sounding measurements:
  Geophysical Prospecting, {\bf 19}, 192--217.
\newblock (\href{http://dx.doi.org/10.1111/j.1365-2478.1971.tb00593.x}{doi:
  10.1111/j.1365-2478.1971.tb00593.x}).

\bibitem[Hamilton, 2000]{RAS.00.Hamilton}
Hamilton, A. J.~S.,  2000, Uncorrelated modes of the non-linear power spectrum:
  Monthly Notices of the Royal Astronomical Society, {\bf 312}, 257--284.
\newblock (\href{http://dx.doi.org/10.1046/j.1365-8711.2000.03071.x}{doi:
  10.1046/j.1365-8711.2000.03071.x}).

\bibitem[Hunziker et~al., 2016]{GEO.16.Hunziker}
Hunziker, J., J. Thorbecke, J. Brackenhoff, and E. Slob,  2016, Inversion of
  controlled-source electromagnetic reflection responses: Geophysics, {\bf 81},
  F49--F57.
\newblock (\href{http://dx.doi.org/10.1190/geo2015-0320.1}{doi:
  10.1190/geo2015-0320.1}).

\bibitem[Hunziker et~al., 2015]{GEO.15.Hunziker}
Hunziker, J., J. Thorbecke, and E. Slob,  2015, The electromagnetic response in
  a layered vertical transverse isotropic medium: {A} new look at an old
  problem: Geophysics, {\bf 80}, F1--F18.
\newblock (\href{http://dx.doi.org/10.1190/geo2013-0411.1}{doi:
  10.1190/geo2013-0411.1}).

\bibitem[Key, 2009]{GEO.09.Key}
Key, K.,  2009, {1D} inversion of multicomponent, multifrequency marine {CSEM}
  data: {M}ethodology and synthetic studies for resolving thin resistive
  layers: Geophysics, {\bf 74}, F9--F20.
\newblock (\href{http://dx.doi.org/10.1190/1.3058434}{doi: 10.1190/1.3058434}).

\bibitem[Key, 2012]{GEO.12.Key}
--------, 2012, Is the fast {H}ankel transform faster than quadrature?:
  Geophysics, {\bf 77}, F21--F30.
\newblock (\href{http://dx.doi.org/10.1190/GEO2011-0237.1}{doi:
  10.1190/GEO2011-0237.1}).

\bibitem[Kong, 2007]{GP.07.Kong}
Kong, F.~N.,  2007, Hankel transform filters for dipole antenna radiation in a
  conductive medium: Geophysical Prospecting, {\bf 55}, 83--89.
\newblock (\href{http://dx.doi.org/10.1111/j.1365-2478.2006.00585.x}{doi:
  10.1111/j.1365-2478.2006.00585.x}).

\bibitem[Løseth and Ursin, 2007]{GJI.07.Loseth}
Løseth, L.~O., and B. Ursin,  2007, Electromagnetic fields in planarly layered
  anisotropic media: Geophysical Journal International, {\bf 170}, 44--80.
\newblock (\href{http://dx.doi.org/10.1111/j.1365-246X.2007.03390.x}{doi:
  10.1111/j.1365-246X.2007.03390.x}).

\bibitem[Nekut and Spies, 1989]{PIEEE.89.Nekut}
Nekut, A.~G., and B.~R. Spies,  1989, Petroleum exploration using
  controlled-source electromagnetic methods: Proceedings of the IEEE, {\bf 77},
  338--362.
\newblock
  (\href{http://ieeexplore.ieee.org/lpdocs/epic03/wrapper.htm?arnumber=18630}{doi:
  10.1109/5.18630}).

\bibitem[Shanks, 1955]{JMP.55.Shanks}
Shanks, D.,  1955, Non-linear transformations of divergent and slowly
  convergent sequences: Journal of Mathematics and Physics, {\bf 34}, 1--42.
\newblock (\href{http://dx.doi.org/10.1002/sapm19553411}{doi:
  10.1002/sapm19553411}).

\bibitem[Slob et~al., 2010]{PIER.10.Slob}
Slob, E., J. Hunziker, and W.~A. Mulder,  2010, Green's tensors for the
  diffusive electric field in a {VTI} half-space: PIER, {\bf 107}, 1--20.
\newblock (\href{http://dx.doi.org/10.2528/PIER10052807}{doi:
  10.2528/PIER10052807}).

\bibitem[Wait, 1982]{B.AP.82.Wait}
Wait, J.~R.,  1982, Geo-{E}lectromagnetism: Academic Press Inc.
\newblock ({I}SBN: 978-0127308807).

\bibitem[Wynn, 1956]{MC.56.Wynn}
Wynn, P.,  1956, {On a device for computing the $e_m(S_n)$ tranformation}:
  Math. Comput., {\bf 10}, 91--96.
\newblock (\href{http://dx.doi.org/10.1090/S0025-5718-1956-0084056-6}{doi:
  10.1090/S0025-5718-1956-0084056-6}).

\bibitem[Ziolkowski and Wright, 2012]{IEEE.12.Ziolkowksi}
Ziolkowski, A., and D. Wright,  2012, The potential of the controlled source
  electromagnetic method: A powerful tool for hydrocarbon exploration,
  appraisal, and reservoir characterization: Signal Processing Magazine, IEEE,
  {\bf 29}, 36--52.
\newblock (\href{http://dx.doi.org/10.1109/MSP.2012.2192529}{doi:
  10.1109/MSP.2012.2192529}).

\end{thebibliography}

\end{document}
